\documentclass[a4paper,12pt]{article}

% Sprache & Zeichensätze
\usepackage[ngerman]{babel}
\usepackage[utf8]{inputenc}
\usepackage[T1]{fontenc}
\usepackage{lmodern}
\usepackage{amsmath}
\usepackage{forest}

% Layout
\usepackage[a4paper,top=2.5cm,bottom=2.5cm,left=2.5cm,right=2.5cm]{geometry}
\setlength{\parindent}{0pt}
\setlength{\parskip}{6pt}

\begin{document}

% Kopfzeile
Hendrik Atzler \hfill Mtk: 88459

\rule{\textwidth}{0.4pt}

{\textbf{2. Übung im Modul „Modellierung“}\hfill Wintersemester 2025/26} \\
zu lösen bis 22. Oktober 2025 \\
\rule{\textwidth}{0.4pt}

{\large{\textbf{Aufgabe 2.1}}} \\[1em]

%Aufgaben------------------------------------------------------
% -------------------------------------------------------------
\textbf{1) } $\neg\neg p$ \\[0.3em]
\textbf{Formel:} Ja \\
\textbf{Begründung:} Jeder Junktor ($\neg$ ist einstellig) ist gebunden.\\
\textbf{Variablen:} \{ $p$ \} \\
\textbf{Teilformeln:} \{ $\neg \neg p, \neg p, p$ \} \\[0.3em]
\begin{center}
\begin{forest}
for tree={inner sep=1pt, l=1cm, s sep=8pt, if n children=0{tier=leaf}{}}
[$\neg$
  [$\neg$
    [$p$]
  ]
]
\end{forest}
\end{center}

% -------------------------------------------------------------
\textbf{2) } $\neg p \lor \land q$ \\[0.3em]
\textbf{Formel:} Nein \\
\textbf{Begründung:} $\land$ und $\lor$ sind zweistellige Junktoren.\\

% -------------------------------------------------------------
\textbf{3) } $p \implies (\neg p \lor ((\neg\neg q) \implies (p \land q)))$ \\[0.3em]
\textbf{Formel:} Ja \\
\textbf{Begründung:} Jeder Junktor ist gebunden.\\
\textbf{Variablen:} \{ $p, q$ \} \\
\textbf{Teilformeln:} \{ $p \implies (\neg p \lor ((\neg\neg q) \implies (p \land q)))$,
$\neg p \lor ((\neg\neg q) \implies (p \land q))$,
$\neg p$,
$(\neg\neg q) \implies (p \land q)$,
$\neg\neg q$,
$\neg p$,
$p \land q$,
$p$,
$q$ \} \\[0.3em]
\begin{center}
\begin{forest}
for tree={inner sep=1pt, l=1.2cm, s sep=8pt, if n children=0{tier=leaf}{}}
[$\implies$
  [$p$]
  [$\lor$
    [$\neg$ [$p$]]
    [$\implies$
      [$\neg$ [$\neg$ [$q$]]]
      [$\land$ [$p$] [$q$]]
    ]
  ]
]
\end{forest}
\end{center}

% -------------------------------------------------------------
\textbf{4) } $p \implies (\neg p \lor (\neg p\neg))$ \\[0.3em]
\textbf{Formel:} Nein \\
\textbf{Begründung:} $\neg$ ist ein einstelliger Junktor aber ungebunden.\\

% -------------------------------------------------------------
\textbf{5) } $(p \implies q) \land (\neg r \implies (q \lor (\neg p \lor r)))$ \\[0.3em]
\textbf{Formel:} Ja \\
\textbf{Begründung:} Jeder Junktor ist gebunden.\\
\textbf{Variablen:} \{ $p, q, r$ \} \\
\textbf{Teilformeln:} \{ $(p \implies q) \land (\neg r \implies (q \lor (\neg p \lor r)))$, 
$p \implies q$, 
$\neg r \implies (q \lor (\neg p \lor r))$,
$\neg r$, 
$q \lor (\neg p \lor r)$,  
$q$,
$\neg p \lor r$,  
$\neg p$, 
$p$, 
$r$ \} \\[0.3em]
\begin{center}
\begin{forest}
for tree={inner sep=1pt, l=1.2cm, s sep=8pt, if n children=0{tier=leaf}{}}
[$\land$
  [$\implies$ [$p$] [$q$]]
  [$\implies$
    [$\neg$ [$r$]]
    [$\lor$
      [$q$]
      [$\lor$ [$\neg$ [$p$]] [$r$]]
    ]
  ]
]
\end{forest}
\end{center}

% -------------------------------------------------------------
\textbf{6) } $p \implies (((q \land \neg r) \implies q) \lor (\neg p \lor r))$ \\[0.3em]
\textbf{Formel:} Ja \\
\textbf{Begründung:} Jeder Junktor ist gebunden.\\
\textbf{Variablen:} \{ $p, q, r$ \} \\
\textbf{Teilformeln:} \{ $p \implies (((q \land \neg r) \implies q) \lor (\neg p \lor r))$, 
$((q \land \neg r) \implies q) \lor (\neg p \lor r)$, 
$(q \land \neg r) \implies q$,
$q \land \neg r$,   
$q$, 
$\neg r$, 
$\neg p \lor r$, 
$\neg p$, 
$p$,
$r$ \} \\[0.3em]
\begin{center}
\begin{forest}
for tree={inner sep=1pt, l=1.2cm, s sep=8pt, if n children=0{tier=leaf}{}}
[$\implies$
  [$p$]
  [$\lor$
    [$\implies$
      [$\land$ [$q$] [$\neg$ [$r$]]]
      [$q$]
    ]
    [$\lor$ [$\neg$ [$p$]] [$r$]]
  ]
]
\end{forest}
\end{center}

% -------------------------------------------------------------
\textbf{7) } $q\neg \land r \implies r$ \\[0.3em]
\textbf{Formel:} Nein \\
\textbf{Begründung:} $\neg$ ist einstelliger Junktor und $\land$ ist ein zweistelliger Junktor, bei sind zu wenig angebunden.\\

% -------------------------------------------------------------
\textbf{8) } $(\neg(\neg p \land \neg q) \lor r) \implies (p \land \neg(\neg q \lor \neg r))$ \\[0.3em]
\textbf{Formel:} Ja \\
\textbf{Begründung:} Jeder Junktor ist gebunden.\\
\textbf{Variablen:} \{ $p, q, r$ \} \\
\textbf{Teilformeln:} \{ $(\neg(\neg p \land \neg q) \lor r) \implies (p \land \neg(\neg q \lor \neg r))$, 
$\neg(\neg p \land \neg q) \lor r$, 
$\neg(\neg p \land \neg q)$,
$\neg p \land \neg q)$,
$\neg p$,
$p \land \neg(\neg q \lor \neg r)$,
$p$,   
$\neg(\neg q \lor \neg r)$,
$\neg q \lor \neg r$,  
$\neg q$, 
$\neg r$, 
$q$, $r$ \} \\[0.3em]
\begin{center}
\begin{forest}
for tree={inner sep=1pt, l=1.2cm, s sep=8pt, if n children=0{tier=leaf}{}}
[$\implies$
  [$\lor$
    [$\neg$ [$\land$ [$\neg$ [$p$]] [$\neg$ [$q$]]]]
    [$r$]
  ]
  [$\land$
    [$p$]
    [$\neg$ [$\lor$ [$\neg$ [$q$]] [$\neg$ [$r$]]]]
  ]
]
\end{forest}
\end{center}

% -------------------------------------------------------------
\textbf{9) } $\neg(\neg t \land \neg p) \leftrightarrow (f \implies q)$ \\[0.3em]
\textbf{Formel:} Ja \\
\textbf{Begründung:} Jeder Junktor ist gebunden.\\
\textbf{Variablen:} \{ $t, p, f, q$ \} \\
\begin{center}
\begin{forest}
for tree={inner sep=1pt, l=1.2cm, s sep=8pt, if n children=0{tier=leaf}{}}
[$\leftrightarrow$
  [$\neg$ [$\land$ [$\neg$ [$t$]] [$\neg$ [$p$]]]]
  [$\implies$ [$f$] [$q$]]
]
\end{forest}
\end{center}

% -------------------------------------------------------------
\vspace{1em}
\textbf{Zusammenfassung:}\\
Gültige aussagenlogische Formeln: 1, 3, 5, 6, 8, 9 \\
Keine aussagenlogischen Formeln: 2, 4, 7

\end{document}
